\documentclass{llncs}
\usepackage{times}
\usepackage[T1]{fontenc}
\usepackage[utf8]{inputenc}
\usepackage[portuges]{babel}

\usepackage{a4}
%\usepackage[margin=3cm,nohead]{geometry}
\usepackage{fancyvrb}
\usepackage{amsmath}
\usepackage{float}
\usepackage{wrapfig}
\usepackage[pdftex]{hyperref}

\begin{document}
%
\title{Smart Cities - Network Infrastructures, Protocols and Services}
%
%\titlerunning{Abbreviated paper title}
% If the paper title is too long for the running head, you can set
% an abbreviated paper title here
%
\author{Joana Alves\orcidID{a93290} \and
João Machado \orcidID{a89510} \and
Rui Armada\orcidID{a90468}}

\institute{ Redes de Computadores \\ Departamento de Informática \\ Universidade do Minho }

\maketitle              % typeset the header of the contribution
%
\begin{abstract}
    \par Neste trabalho, no âmbito da unidade curricular de Redes de Computadores,
    pretende-se desenvolver um tema relacionado com a área de Redes, tendo o grupo escolhido 
    o tema: \textit{Smart Cities} (cidades inteligentes). Para isso, foi necessária a pesquisa
    e recolha de informação credível e viável a partir de artigos científicos verificados, tendo alguns destes sido cedidos pelos docentes.
    Assim, apresentamos neste relatório um resumo de toda a pesquisa obtida sobre o tema,
    assim como a sua infraestrutura e serviços.

\keywords{Smart Cities \and IoT.}
\end{abstract}
%
%
%
\paragraph{}
\paragraph{}
\section{Introdução}
    \par A partir dos conceitos \textbf{IoT} (\textit{Internet of Things}) - "um paradigma de comunicação que tem como objetivo final conectar uma infinidade de dispositivos digitais à Internet" \cite{ref_article1} - foi desenvolvido o conceito de \textit{smart cities}, alvo de estudo deste relatório.
    \par Como tal, para começar, é necessário saber em concreto o que é uma \textit{Smart City}. Uma \textit{smart city} é capaz de utilizar tecnologias de informação e comunicação \textit{ICT} para melhorar a sua eficiencia operational, partilha de informação com o público e oferecer uma melhor qualidade governamental, de serviço e de bem estar do cidadão.
    \par Assim, conseguimos entender que uma \textit{smart city} é, de facto, uma consequência
    de IoT, uma vez que a partir das premissas de conexão à Internet de IoT conseguimos atingir o 
    objetivo final de construção de cidades inteligentes.
    \par Neste relatório vamos apresentar definições mais pormenorizadas do tema, assim
    como as suas particularidades, infraestrutura e desafios.
    
\newpage
    
    
    
    
\section{Smart Cities}

    \par Uma \textit{smart city} é um complexo 
    ecossistema urbano caracterizado pelo uso intenso de informação e tecnologias de comunicação \textit{ICT} com o objetivo de aumentar a sua autonomia, eficiência e sustentabilidade, criando assim um espaço único para inovação e empreendedorismo. 
    As principais partes interessadas incluem \textit{developers} de aplicações, fornecedores de serviços, cidadãos, governo e prestadores de serviços públicos, a comunidade de investigação e os criadores de plataformas.
    \par As aplicações de \textit{smart cities} que têm por base IoT podem ser 
    categorizadas com base no tipo de rede, escalabilidade, cobertura,
    flexibilidade, heterogeneidade e envolvimento do
    utilizador final. Em geral, estas aplicações podem ser agrupadas
    em pessoais e domésticas, utilitárias, móveis e empresas.
    
    \subsubsection{Pessoais e Domésticas:} As aplicações pessoais e domésticas incluem serviços de saúde eletrónicos omnipresentes que funcionam independentemente através de redes de área corporal (BANs), que ajudam os médicos a monitorizar os pacientes remotamente.
    
    \subsubsection{Utilitários:} As aplicações utilitárias incluem \textit{smart
    grid}, monitorização inteligente, monitorização de redes de água, e 
    vigilância por vídeo.
    
    \subsubsection{Móveis:} As aplicações móveis incluem sistema de transporte
    inteligente (ITS), gestão de tráfego, controlo de congestionamento
    e gestão de resíduos. 
    
    \paragraph{}
    \par Foram feitos vários esforços de investigação para integrar IoT com
    ambientes de \textit{smart city}. No entanto, ainda não se chegou a um consenso
    sobre qual a melhor forma de visualizar ou implementar o conceito. Assim, 
    podemos dividir o conceito nas suas múltiplas partes integrantes e discutir cada uma delas. 
    
    
    \subsection{Protocolos de Comunicação}
    \par Uma cidade inteligente baseada em IoT depende de inúmeros protocolos de
    comunicação de curto e longo alcance 
    para transportar dados entre dispositivos e servidores \textit{back-end}. 
    \par As tecnologias sem fios \textit{short-range} mais relevantes incluem 
    Bluetooth, Wi-Fi, WiMAX, e IEEE 802.11p, que são utilizados principalmente em
    cuidados de saúde electrónicos e comunicação veicular. 
    Tecnologias de longo alcance como o \textit{Global System for Mobile 
    Communication} (GSM), \textit{Long Term Evolution} (LTE), e 
    \textit{LTE-Advanced} são normalmente utilizados em ITS tais como 
    \textit{e-healthcare} e \textit{smart grid}. 
    \par Para além dos protocolos mais conhecidos conhecidos, 
    a \textit{LoRa Alliance} está a normalizar o protocolo LoRaWAN para apoiar
    aplicações de cidades inteligentes, 
    assegurando principalmente a interoperabilidade entre vários operadores. 

    
    \subsection{Fornecedores de Serviços}
    IoT é reconhecida como potencial de aumento de receitas dos
    prestadores de serviços. Assim, os prestadores de serviços mundialmente 
    conhecidos começaram já a explorar este paradigma de comunicação.
    Os principais prestadores de serviços como Nokia e Vodafone, oferecem uma
    variedade de serviços e plataformas para aplicações de cidades inteligentes
    tais como ITS e logística, \textit{smart metering}, automação de casas e \textit{e-healthcare}.

    
    \subsection{Tipos de Redes}
    \par As aplicações de \textit{smart cities} baseadas em IoT dependem 
    de várias topologias de rede para funcionar num ambiente totalmente autónomo.
    As redes IoT oferecem serviços de curto alcance, como por exemplo redes locais sem fios (WLANs),
    BANs, e redes de área pessoal sem fios (WPANs). 
    \par As áreas de aplicação incluem
    serviços de \textit{e-healthcare} no interior e iluminação pública. Por outro lado, as
    aplicações tais como ITS, \textit{e-healthcare} móvel e tratamento de resíduos
    utilizam redes de área ampla (WANs), redes da área metropolitana (MANs) e redes de comunicação
    móveis. 


    
    \subsection{Organizações Padrão}
    As aplicações de \textit{smart city} não só exigem implementação em larga escala de
    vários tipos de dispositivos IoT, mas também exigem a interoperabilidade dos dispositivos.
    Por conseguinte, os órgãos directivos mais proeminentes tais como a \textit{Internet Engineering
    Task Force} (IETF), Projecto de Parceria de Terceira Geração (3GPP), Instituto Europeu de Normas
    de Telecomunicações (ETSI), IEEE e Open Mobile Alliance (OMA) estão envolvidos no
    desenvolvimento de normas para apoiar aplicações de \textit{smart city} numa grande escala.
    
    \subsubsection{IETF:} 
    O primeiro grupo de trabalho (6LoWPAN) normalizou técnicas para manipulação de pequenos
    pacotes IoT usando compressão de cabeçalho e otimização de descoberta de vizinhos.
    Além disso, o grupo de trabalho \textit{Routing Over Low-power and Lossy network} desenvolveu
    protocolos de \textit{routing} para aplicações de \textit{smart cities}. 
    
    \subsubsection{3GPP:}
    Normalizou a banda estreita (\textit{narrowband}) IoT (NB-IoT) para fornecer uma melhor cobertura 
    de rede para aplicações de \textit{smart cities}. Como resultado, NB-IoT cumpre os requisitos de aplicação
    nos domínios industrial, público, pessoal e doméstico. Adicionalmente, a 3GPP está a introduzir técnicas 
    de transmissão/recepção descontínua alargada (eDTX/eDRX) na versão 13 para reduzir ainda mais o consumo
    de energia, e assim aumentar o tempo de funcionamento do dispositivo.
    
    \subsubsection{ETSI:}
    Tem como objetivo fornecer soluções interoperáveis e rentáveis para apoiar aplicações de cidades inteligentes.
    Particularmente, \textit{oneM2M} é a iniciativa global do ETSI para apoiar a conectividade de IoT em larga escala.
    A iniciativa \textit{oneM2M} visa desenvolver uma plataforma horizontal única para permitir a
    interoperabilidade entre todas as aplicações através de uma camada de software distribuído. Além disso, fornece arquitectura, requisitos, \textit{APIs} e soluções de privacidade e segurança. Consequentemente, as \textit{APIs} e interfaces abertas podem ser utilizadas dentro de vários sistemas para permitir uma infinidade de ligações entre dispositivos IoT e servidores \textit{backend}.
    
    \subsubsection{IEEE:} 
    No contexto das cidades inteligentes, concentra-se principalmente na optimização da interface aérea
    para destacamentos ultra-IoT. Além disso, concentra-se sobre a utilização do espectro de sub-6 GHz para a conectividade IoT para suportar várias aplicações de cidades inteligentes.
    
    \subsubsection{OMA:} 
    Normalizou o protocolo \textit{OMA Lightweight M2M} (OMALWM2M) para gestão de dispositivos IoT com
    recursos limitados tanto para redes de sensores como redes celulares. OMALWM2M oferece um caminho de comunicação
    entre um cliente e um servidor (ambos LWM2M). Por conseguinte, é um protocolo leve e compacto que é
    frequentemente utilizado com o Protocolo de Aplicação Restrita (CoAP), e oferece um recurso eficiente de
    dados modelo para os dispositivos IoT com recursos limitados. Além disso, proporciona uma escolha para os
    fornecedores de serviços para a implantação de sistemas IoT de apoio às aplicações de cidades
    inteligentes correspondentes.
    
    
    
\section{Desafios}
Com todos os dispositivos conectados via IoT, as cidades inteligentes procurarão
tornar a vida mais fácil e mais habitável. No entanto, quanto mais nos
aproximamos da implementação dessa visão, mais a procura
de desenvolvimento de tecnologia inteligente, de uma rede mais sólida
e de profissionais de TI qualificados aumenta. Para além disto, o desenvolvimento e implementação de uma cidade inteligente acarreta sempre alguns desafios sociais, económicos e/ou relativos à privacidade e segurança dos cidadãos.

    \subsection{Privacidade e Segurança}
    \par O principal motivo de preocupação é o uso indevido da privacidade
    uma vez que nem todos os dispositivos conectados são ciber-resilientes.
    Numa era em que tudo está conectado, desde a torradeira até ao 
    frigorífico ou automóvel, torna-se imperativo o desenvolvimento de um 
    sistema à prova de pirataria.
    \par Outra preocupação centra-se no facto de poder incutir nos
    cidadãos o sentimento de estarem a ser vigiados a 
    toda a hora. Câmaras e sensores de vigilância instalados em cada 
    canto da cidade podem estar lá por uma boa causa, mas vão 
    definitivamente gerar desconforto em alguns cidadãos, o que leva
    à questão: será que os benefícios valem realmente a pena tendo em conta
    o preço que temos a pagar a nível de privacidade?
    
    \subsection{Infraestrutura}
    \par As cidades inteligentes utilizam tecnologia de sensores para 
    recolher e analisar informação, recolhendo dados como estatísticas de
    hora de ponta, taxas de criminalidade ou qualidade geral do ar. A 
    instalação e manutenção destes sensores envolve infraestruturas
    complicadas e dispendiosas. Como serão alimentados? Envolverão fios 
    rígidos, energia solar ou funcionamento por bateria?
    \par As grandes áreas metropolitanas já são desafiadas com a substituição de infraestruturas com décadas de existência, tais como cabos subterrâneos e túneis de transporte, bem como com a instalação de Internet de alta velocidade. O serviço sem fios de banda larga está a aumentar, mas ainda existem áreas nas grandes cidades onde o acesso é limitado. 
    \par O financiamento para novos projectos de infraestruturas é limitado e os processos de aprovação podem demorar anos. A instalação de novos sensores e outras melhorias causam problemas temporários - embora ainda frustrantes - às pessoas que vivem nestas cidades.
    
    
    \subsection{Inclusão Social}
    \par Programas de trânsito inteligentes que dão aos condutores atualizações em tempo real são uma grande ideia para uma cidade movimentada. Mas e se metade da população dessa cidade não puder dar-se ao luxo de utilizar trânsito em massa (transportes que albergam um elevado número de pessoas, como autocarros ou comboios) ou Uber? E uma população cada vez mais envelhecida que não utiliza dispositivos móveis ou aplicações? Como é que a tecnologia inteligente irá alcançar e beneficiar estes grupos de pessoas?
    \par É vital que o planeamento da cidade inteligente envolva a consideração de todos os grupos de pessoas, e não apenas dos economica e tecnologicamente avançados. A tecnologia deve estar sempre a trabalhar para aproximar as pessoas em vez de as dividir ainda mais com base no rendimento ou nos níveis de educação.
    
    
    
    
    
\newpage
\section{Conclusão}
    \par Embora a maioria possa concordar que a tecnologia inteligente tem o poder de tornar as nossas vidas muito mais simples - especialmente em áreas urbanas - a implementação dessa tecnologia deve ser feita de forma cuidadosa, planeada e segura. Em vez de se concentrarem apenas no que a solução pode fazer, os promotores e as empresas de tecnologia devem também considerar como é que ela irá afectar as pessoas que entram em contacto com ela.
    \par Quando a tecnologia, a administração da cidade e as comunidades de pessoas se juntam para melhorar a qualidade de vida de todos os envolvidos, é aí que uma cidade se torna verdadeiramente "inteligente".
    
    
    
    
\mbox{}
\vfill
\begin{thebibliography}{8}
\bibitem{ref_article1}
Mehmood, Y., Ahmad, F., Yaoob, I., Adnane, A., Imran, M., Guizani, S.: Internet-of-Things-Based Smart Cities: Recent Advances and Challenges. IEEE Communications Magazine, (2017)


\bibitem{ref_url1}
Security, Privacy and Risks Within Smart Cities: Literature Review and Development of a Smart City Interaction Framework, \url{https://link.springer.com/article/10.1007/s10796-020-10044-1}. Last accessed 28 Fev 2022


\bibitem{ref_url2}
Internet of Things for Smart Cities, \url{https://ieeexplore.ieee.org/abstract/document/6740844}. Last accessed 28 Fev 2022



\end{thebibliography}
\end{document}
